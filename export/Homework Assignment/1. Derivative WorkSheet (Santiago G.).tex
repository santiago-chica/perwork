\documentclass[12pt]{article}%
\usepackage[T1]{fontenc}%
\usepackage[utf8]{inputenc}%
\usepackage{lmodern}%
\usepackage{textcomp}%
\usepackage{lastpage}%
\usepackage{geometry}%
\geometry{margin=1in}%
\usepackage{enumerate}%
\usepackage{amsmath}%
\usepackage{enumitem}%
\usepackage{fancyhdr}%
%
\fancypagestyle{header}{%
\renewcommand{\headrulewidth}{0.2pt}%
\renewcommand{\footrulewidth}{0.2pt}%
\fancyhead{%
}%
\fancyfoot{%
}%
\fancyhead[L]{%
Santiago G.%
}%
\fancyhead[C]{%
Derivative WorkSheet%
}%
\fancyhead[R]{%
11/11/2024%
}%
\fancyfoot[C]{%
\thepage%
}%
}%
\setlength{\headheight}{15pt}%
%
\begin{document}%
\normalsize%
\pagestyle{header}%
Esto es un texto de prueba...%
\begin{enumerate}[wide, labelwidth=!, labelindent=0pt,label={Pregunta \arabic*. }]%
\item%
%
A continuacion la definicion de la integral por partes%
\[\int udv=uv-\int vdu\]%
Basado en eso, resuelva las siguientes operaciones%
\begin{enumerate}[wide, labelwidth=!, labelindent=0pt,label={\Alph*) }]%
\item%
%
Integral definida\\%
\item%
%
\(\int xcos(x)dx\)%
\item%
%
\(\int xcos(x)dx\)%
\item%
%
\(\int xcos(x)dx\)%
\end{enumerate}%
\[ x^n + y^n = z^n \]%
\item%
%
Resuelva la ecuacion.%
\begin{equation*}5 x = 6\end{equation*}%
\begin{equation*}\frac{6}{5}\end{equation*}%
\item%
%
Resuelva la ecuacion.%
\begin{equation*}1 + 10\end{equation*}%
\begin{equation*}11\end{equation*}%
\item%
%
Resuelva la ecuacion.%
\begin{equation*}7 - 8\end{equation*}%
\begin{equation*}-1\end{equation*}%
\item%
%
Resuelva la ecuacion.%
\begin{equation*}- 5 x\end{equation*}%
\begin{equation*}-5\end{equation*}%
\item%
%
Resuelva la ecuacion.%
\begin{equation*}7 x^{2}\end{equation*}%
\begin{equation*}14 x\end{equation*}%
\item%
%
Resuelva la ecuacion.%
\begin{equation*}6 x^{4}\end{equation*}%
\begin{equation*}24 x^{3}\end{equation*}%
\item%
%
Resuelva la ecuacion.%
\begin{equation*}- x^{5}\end{equation*}%
\begin{equation*}- 5 x^{4}\end{equation*}%
\item%
%
Resuelva la ecuacion.%
\begin{equation*}2 x^{4}\end{equation*}%
\begin{equation*}8 x^{3}\end{equation*}%
\item%
%
Resuelva la ecuacion.%
\begin{equation*}6 x^{2}\end{equation*}%
\begin{equation*}12 x\end{equation*}%
\item%
%
Resuelva la ecuacion.%
\begin{equation*}10 x^{5}\end{equation*}%
\begin{equation*}50 x^{4}\end{equation*}%
\item%
%
Resuelva la ecuacion.%
\begin{equation*}- 2 x^{2}\end{equation*}%
\begin{equation*}- 4 x\end{equation*}%
\item%
%
Resuelva la ecuacion.%
\begin{equation*}8 x^{3}\end{equation*}%
\begin{equation*}24 x^{2}\end{equation*}%
\item%
%
Resuelva la ecuacion.%
\begin{equation*}9 x^{3}\end{equation*}%
\begin{equation*}27 x^{2}\end{equation*}%
\end{enumerate}%
\end{document}